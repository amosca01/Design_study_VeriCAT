\section{Related Work}

Despite the fact that advanced MT models typically include extremely opaque elements (ex. RNNs), and that their output is meant to be used in human-AI collaborative tasks\cite{mauvcec2019machine}, there has been very little research on the usability of MT output and how XAI may help. Our work centers on the meeting of these two fields; namely in identifying how XAI may be used to improve the utility of MT output. To set the stage for out work we provide high-level summaries of existing work in MT and XAI in the following sections.  

\subsection{Machine Translation Quality Estimation} 

The study of MT began in the 1950's, but it was not until recently that significant advances were made which greatly improved accuracy and usability of MT. Despite these advances, human translators still outperform MT in terms of accuracy and preserving the original meaning of translated text\cite{mauvcec2019machine}. As use of MT becomes more widespread, problems can arise when translations are inaccurate. One way to address this problem is with a secondary machine learning model, which predicts the quality of the MT model's output.  

There are many ways to evaluate MT quality. Numerous automatic metrics aim to approximate human judgement. Some common examples include BLEU, NIST, METEOR, and TER. Additionally, there are some human in the loop automatic judgements. For instance HTER (human-mediated translation error rate), which attempts to capture the number of post edits made to a MT by a human translator\cite{mauvcec2019machine}. On the other hand there are human judgements which come from direct assessment (DA) of aspects of the translation such as fluency and adequacy\cite{snover2009Fluency}. Some of these metrics can be used to train a QE model to predict the quality of specific sentences. While certain metrics are only appropriate for document-level quality assessment, HTER and DA have previously been used to train QE models.            

While many advancements have been made in MT QE, there has been limited work studying the usability of these tools to humans, and their impact on users' ability to make decisions based on perceived translation quality. While OpenKiwi \cite{UnBabel} performed a demo of a user interface for QE at ACL 2019, they have not released the code or demo, nor have they released a user study. Further, there is a lack of tools available to make QE accessible to non-MT experts. Avramidis created a GUI to make QE more accessible to non-experts, however users must still be proficient in Python and the command line\cite{avramidis2017QE}.    

At the IUI conference in 2020 researchers presented a demonstration titled: \textit{XAIT: An Interactive Website for Explainable AI for Text}. We used this work as a starting point for investigating XAI efforts for MT, and unfortunately found that none of the cited work related to XAI for MT\cite{oduor2020XAIT}. In fact, the only usability study we were able to uncover for MT was one by Martindale and Carpuat which investigated how revealing to users errors in fluency and adequacy of MT might change their trust in MT. In this work they found that poor fluency in translations can significantly effect users' trust of MT, but that trust is easily rebuilt\cite{martindaleFluency2018}.       

\ab{I propose we take out the XAI section and add a section on MT & Vis}
\subsection{XAI Tools \& Approaches}
Though we were unable to find specific XAI tools for MT, we provide an overview of the general approaches that exist. Today’s XAI tools employ a variety of explanation strategies. For example, tools like LIME \cite{RiberoLIME2016}, use a simpler model to approximate the behavior of a complex model \cite{SelbstBarocasIntuitive2018}. Proponents of this “model of the model” \cite{SelbstBarocasIntuitive2018} approach – which is also sometimes referred to as generating approximate models – argue that it can present complex processes in a way that is understandable, flexible and mostly accurate \cite{MittelstadtRussellExplain2019}. %However, some argue that the approach can be misleading \cite{rudin2018stop} as approximate models may imply a false sense of simplicity, allow for improper conclusions, or be used to bolster predetermined narratives \cite{MittelstadtRussellExplain2019} and \cite{herman2017promise}. 

Other tools, such as TCAV \cite{KimTCAV2018} and SHAP \cite{LundbergLeeSHAP2017}, use a different explanation strategy, helping users build intuition about how models work by allowing them to test and explore how different inputs relate to different outputs. Rather than explaining the internal rules of a model, these tools purport to help users determine which factors contributed (most) to a particular output \cite{SelbstBarocasIntuitive2018}. %This approach can help users reason about why a model might have made a specific prediction. For example, “reason codes” — text-based explanations about the importance of different input variables — are common in the context of lending \cite{SelbstBarocasIntuitive2018}. However, critics of this “input-output” approach warn that it ceases to be useful when a model’s output results from complex interactions between many factors \cite{SelbstBarocasIntuitive2018}. Additionally, some warn that analysis of factors contributing to a specific output can lead to improper conclusions about the model’s behavior overall \cite{DoshiVelezAccountability2017}. 

A third explanation strategy advocates for the use of simple models that are naturally more interpretable \cite{rudin2018stop}, at least in cases where interpretability is paramount. These types of models are sometimes referred to as “white box”. % and some of DARPA’s XAI efforts can be categorized under this “design for simplicity” approach \cite{DARPAXAI}. Critics, however, invoke a commonly-cited tradeoff between complexity and accuracy, questioning whether designing for simplicity is possible without a loss in predictive accuracy.

Notably, there is a method missing for our use case. None of the typical techniques presented above would be applicable to MT. We postulate that when utility is the goal, the best way to explain an opaque MT model may be by adding contextual information via a QE model. This method differs from typical methods because it focuses on providing information to \textit{improve the utility} of AI output, as opposed providing information to make users more expert in what is happening under the hood. This is the guiding principle behind VeriCAT.   

\subsection{Visualization and Machine Translation}

Though MT is becoming more widespread, there is a lack of tools designed to help users of MT text understand its reliability. To our knowledge, at the time of writing this paper there was only one tool besides ours built to meet this need. Collins et al. developed a lattice visualization to illustrate uncertainty in MT text. While Collins et al. demonstrate that this method is effective for an instant messaging scenario~\cite{collins2007lattices}, it would not scale well to our use case which includes scanning full passages of text for poor quality translations.   

In other veins, Albrecht et al. developed a human-AI collaborative system that uses visualization to help users gain an intuition about a translation's source language so that they can correct errors in MT text~\cite{albrecht2009chinese}. 
DeNeefe et al. developed an interactive translation
visualization tool called a DerivTool, which is intended to give users intuition about the MT model itself~\cite{deneefe2005interactively}. In contrast to these approaches we are not interested in helping users develop intuition about the source language of translated text, or about the MT model itself. Instead, we are focused on providing users with contextual information about when and whether they should trust a particular snippet of MT text. 

%An area in which there is particular overlap between visualization and MT is the Digital Humanities. In this setting, the goal of visualization and MT is typically to enable comparisons of one text in several languages~\cite{janicke2017visual}. For example, ShakerVis is a tool for comparing different translations of \textit{Othello}~\cite{geng2015shakervis}. This work is similar to our in that the goal is making MT text more usable to users, however it differs significantly in  

%Our work differs from  