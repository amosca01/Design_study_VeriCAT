
\section{Motivation}
\label{sec:design_requirements}

%Machine translation is becoming more widespread for a variety of uses. 
Today, finding machine translated text can be as simple as a Google search. However, as revealed by our review of the literature, there are limited tools that communicate to users when, and whether, they should trust MT text.  

%It is not uncommon for analysts to use MT text to scan text in languages they do not speak for potential threats. This is because although human translators outperform MT models, they are in high demand and expensive~\cite{mauvcec2019machine}. However, when analysts do not speak the original language of the text they are asked to analyze, they become %completely \remco{strong word -- hard to defend} 
%dependant on MT. Even if an analyst is aware that MT is not perfect, 

If a user of MT text is not a speaker of the source language and does not have additional contextual information, then their only basis for judging translation quality is fluency. Fluency refers to how well the text follows the target language's norms, taking into account grammar and clarity~\cite{mauvcec2019machine}.  

In many cases, fluency is a reasonable proxy for translation quality. However, it can also fail and the ``attack them" incident is a clear example. As a phrase, ``attack them" sounds fluent despite its very low quality translation. Another example is one we include in our user study. The Russian $\rightarrow$ English FairSeq model we use in this work is inaccurate when translating text in all caps. The model translates all caps Russian text that should read ``(BUT THEY DIDN'T HEAR IT)" to ``(BUT THIS HAPPENED)." While fluent, this translation is low quality because it obscures the meaning of the original text. We design VeriCAT to help users overcome these situations, where fluency is not an adequate proxy for translation quality.

%Similar to Collins et al.~\cite{collins2007lattices}, 
%Our goal was to design a system that provides contextual information for \textit{individual snippets} of translated text, as opposed to the quality of the MT model as a whole. Thus, VeriCAT uses a QE model to provide users with sentence-level estimates of MT quality.

%\begin{compacthang}
%\item \textbf{DR1:} The tool should support Russian text translated into English \remco{each of these should be elaborated} \remco{Note that for this particular DR, if you had mentioned the current tools used by the analysts, this DR might not be relevant because this is a given?}. 

%\item \textbf{DR:} The explanations should help users asses the quality of the translated text and quickly determine when a translation should to be verified by a human translator. 

%\item \textbf{DR3:} The explanations should be accessible to domain experts without expertise in AI or MT. 

%\item \textbf{DR4:} The tool should require little-to-no training to use. 

%\end{compacthang}

  