%%
%% This is file `sample-authordraft.tex',
%% generated with the docstrip utility.
%%
%% The original source files were:
%%
%% samples.dtx  (with options: `authordraft')
%% 
%% IMPORTANT NOTICE:
%% 
%% For the copyright see the source file.
%% 
%% Any modified versions of this file must be renamed
%% with new filenames distinct from sample-authordraft.tex.
%% 
%% For distribution of the original source see the terms
%% for copying and modification in the file samples.dtx.
%% 
%% This generated file may be distributed as long as the
%% original source files, as listed above, are part of the
%% same distribution. (The sources need not necessarily be
%% in the same archive or directory.)
%%
%% The first command in your LaTeX source must be the \documentclass command.
%\documentclass[sigconf,authordraft]{acmart}


%%%% As of March 2017, [siggraph] is no longer used. Please use sigconf (above) for SIGGRAPH conferences.

%%%% As of May 2020, [sigchi] and [sigchi-a] are no longer used. Please use sigconf (above) for SIGCHI conferences.

%%%% Proceedings format for SIGPLAN conferences 
%\documentclass[sigplan, anonymous, authordraft]{acmart}

%%%% Proceedings format for conferences using one-column small layout
%\documentclass[acmsmall,authordraft]{acmart}

% NOTE that a single column version is required for submission and peer review. This can be done by changing the \doucmentclass[...]{acmart} in this template to 
\documentclass[manuscript,screen, anonymous]{acmart}

%%
%% \BibTeX command to typeset BibTeX logo in the docs
\AtBeginDocument{%
  \providecommand\BibTeX{{%
    \normalfont B\kern-0.5em{\scshape i\kern-0.25em b}\kern-0.8em\TeX}}}

%% Rights management information.  This information is sent to you
%% when you complete the rights form.  These commands have SAMPLE
%% values in them; it is your responsibility as an author to replace
%% the commands and values with those provided to you when you
%% complete the rights form.
\setcopyright{acmcopyright}
\copyrightyear{2021}
\acmYear{2021}
\acmDOI{10.1145/1122445.1122456}

%% These commands are for a PROCEEDINGS abstract or paper.
\acmConference[Yokohama '21]{Yokohama '21: ACM CHI Conference on Human Factors in Computing Systems}{May 08--13, 2021}{Yokohama, Japan}
\acmBooktitle{Yokohama '21: ACM CHI Conference on Human Factors in Computing Systems,
  May 08--13, 2021, Yokohama, Japan}
\acmPrice{15.00}
\acmISBN{978-1-4503-XXXX-X/18/06}

%%%%%%%%%%  REMCO'S MACROS  %%%%%%%%%%%%%%%%
\usepackage[normalem]{ulem}
%\usepackage[prologue,dvipsnames,table,xcdraw]{xcolor}
\usepackage{xcolor}
\definecolor{mred}{rgb}{.80,.12,.30}
\definecolor{MRED}{rgb}{.80,.12,.30}
\definecolor{grey}{rgb}{0.5,0.5,0.5}
\definecolor{purple}{rgb}{.75,0,.85}
\definecolor{pistachio}{rgb}{0.58, 0.77, 0.45}

\definecolor{bar-blue}{HTML}{80b1d3}

\definecolor{bar-noXai}{HTML}{8dd3c7}
\definecolor{bar-Qual}{HTML}{ffffb3}
\definecolor{bar-Markup}{HTML}{bebada}
\definecolor{bar-MarkupQ}{HTML}{fb8072}
\definecolor{bar-PredictQ}{HTML}{fdb462}

\newif\ifnotes
\notestrue

\newcommand{\remco}[1]{\ifnotes{\textcolor{mred}{(Remco: #1)}}\fi}
\newcommand{\ab}[1]{\ifnotes{\leavevmode\color{pistachio}{(Ab: #1)}}\else{#1}\fi}
\newcommand{\andrea}[1]{\ifnotes{\textcolor{blue}{(Andrea: #1)}}\else{#1}\fi}
\newcommand{\nina}[1]{\ifnotes{\textcolor{orange}{(Nina: #1)}}\fi}

\let\origcite\cite
\renewcommand{\cite}[1]{\ifnotes\mbox{\origcite{#1}}\else \origcite{#1}\fi}
\newcommand{\strike}[1]{\ifnotes{\color{mred}{\texorpdfstring{\sout{#1}}{#1}}}\fi}
\newcommand{\strikeg}[1]{\ifnotes{\color{grey}{\texorpdfstring{\sout{#1}}{#1}}}\fi}
\newcommand{\add}[1]{\ifnotes{\leavevmode\color{mred}{#1}}\else{#1}\fi}
\newcommand{\replace}[2]{\ifnotes{\strikeg{#1}\add{#2}}\else{#2}\fi}

\newcommand{\cbox}[1]{\raisebox{\depth}{\fcolorbox{black}{#1}{\null}}}

%%%%%%%%%%  END REMCO'S MACROS  %%%%%%%%%%%%%%%%

%%PACKAGES
\usepackage{comment}
\usepackage{float}
\usepackage{arydshln}
\usepackage{threeparttable}
\usepackage{enumitem}
\usepackage{hang}
\usepackage{multirow}
\setlist{nosep}
\usepackage{subcaption}
\usepackage{bchart}
\usepackage[skip=5pt]{caption}
\graphicspath{ {./images/} }



%%
%% Submission ID.
%% Use this when submitting an article to a sponsored event. You'll
%% receive a unique submission ID from the organizers
%% of the event, and this ID should be used as the parameter to this command.
\acmSubmissionID{123-A56-BU3}

%%
%% The majority of ACM publications use numbered citations and
%% references.  The command \citestyle{authoryear} switches to the
%% "author year" style.
%%
%% If you are preparing content for an event
%% sponsored by ACM SIGGRAPH, you must use the "author year" style of
%% citations and references.
%% Uncommenting
%% the next command will enable that style.
%%\citestyle{acmauthoryear}

%%
%% end of the preamble, start of the body of the document source.
\begin{document}

%%
%% The "title" command has an optional parameter,
%% allowing the author to define a "short title" to be used in page headers.
\title[How Good is your Machine Translation?]{How Good is your Machine Translation? Design and Evaluation of a Quality Estimation System}

%%
%% The "author" command and its associated commands are used to define
%% the authors and their affiliations.
%% Of note is the shared affiliation of the first two authors, and the
%% "authornote" and "authornotemark" commands
%% used to denote shared contribution to the research.
\author{Ab Mosca}
\authornote{Both authors contributed equally to this research.}
\email{amosca01@cs.tufts.edu}\affiliation{%
  \institution{Tufts University}
}

\author{Nina Lopatina}
\authornotemark[1]
\email{nlopatina@iqt.org}
\affiliation{%
  \institution{In-Q-Tel Labs}
}

\author{Remco Chang}
\affiliation{\institution{Tufts University}}
\email{remco@cs.tufts.edu}

\author{Andrea Brennen}
\email{abrennen@iqt.org}
\affiliation{%
  \institution{In-Q-Tel Labs}
}

%%
%% By default, the full list of authors will be used in the page
%% headers. Often, this list is too long, and will overlap
%% other information printed in the page headers. This command allows
%% the author to define a more concise list
%% of authors' names for this purpose.
\renewcommand{\shortauthors}{Mosca and Brennen, et al.}

%%
%% The abstract is a short summary of the work to be presented in the
%% article.
\begin{abstract}
  \section{Abstract}

In this paper, we describe the design, development, and evaluation of VeriCAT, a system that predicts the quality of machine translated text. Miss-translation by AI has led to adverse effects in the past, such as erroneous arrests. VeriCAT is designed to address this issue. Short for verification of computer-assisted translation, VeriCAT predicts a sentence-level quality score for individual snippets of Russian text that have been translated into English. The VeriCAT user interface displays these sentence-level quality scores, along with the original and translated text, in order to help users determine whether to trust a specific machine-translated sentence. We evaluate VeriCAT with a quantitative user study to measure how the tool impacts participants’ ability to identify poor quality MTs, and find the tool significantly increases participants’ accuracy on this task. Moreover, we find participants perform the task as accurately with VeriCAT QE scores as with ground truth quality scores.
\end{abstract}

%%
%% The code below is generated by the tool at http://dl.acm.org/ccs.cfm.
%% Please copy and paste the code instead of the example below.
%%
\begin{CCSXML}
<ccs2012>
<concept>
<concept_id>10003120.10003121.10003122.10010854</concept_id>
<concept_desc>Human-centered computing~Usability testing</concept_desc>
<concept_significance>500</concept_significance>
</concept>
<concept>
<concept_id>10003120.10003145.10003147</concept_id>
<concept_desc>Human-centered computing~Visualization application domains</concept_desc>
<concept_significance>500</concept_significance>
</concept>
</ccs2012>
\end{CCSXML}

\ccsdesc[500]{Human-centered computing~Usability testing}
\ccsdesc[500]{Human-centered computing~Visualization application domains}
%%
%% Keywords. The author(s) should pick words that accurately describe
%% the work being presented. Separate the keywords with commas.
\keywords{User-centered design, Machine Translation Quality Estimation}

%% A "teaser" image appears between the author and affiliation
%% information and the body of the document, and typically spans the
%% page.

%\begin{teaserfigure}
%  \includegraphics[width=\textwidth]{sampleteaser.pdf}
%  \caption{Seattle Mariners at Spring Training, 2010.}
%  \Description{Enjoying the baseball game from the third-base
 % seats. Ichiro Suzuki preparing to bat.}
%  \label{fig:teaser}
%\end{teaserfigure}

%%
%% This command processes the author and affiliation and title
%% information and builds the first part of the formatted document.
\maketitle

\section{Introduction}

In 2017 Facebook's machine translation (MT) algorithm incorrectly translated a construction worker's Arabic-language post. The original post said ``good morning" in Arabic, but was erroneously translated into Hebrew as ``attack them", leading to the worker's arrest and several hours of questioning. Notably, no Arabic-speakers were asked to verify the machine translation of the post leading up to the arrest~\cite{hernFacebook2017}. For many users of machine translation, it is easy to forget that translated output is susceptible to error and, as illustrated by this situation, some translation errors can lead to severe consequences. 
%Analysts might use machine translation in situations where human translators are in short supply. To avoid potentially negative consequences of erroneous translations, analysts would benefit from tools that help them  
%asses the quality of specific passages of translated text.  

Our goal in this work is to develop and evaluate a tool to meet this need--helping users determine when, and whether, to trust machine translation. 
Initially, we looked to Explainable AI (XAI) tools and approaches for guidance.
%However, we find that most XAI initiatives are geared towards helping AI experts debug their models, as opposed to helping subject matter experts effectively and efficiently use model output~\cite{brennen2020What}. As such, most state of the art XAI solutions are not particularly helpful for analysts without substantial expertise in AI \remco{the last two sentences are difficult to defend -- a reviewer just needs one counter example, however shitty, to discount the motivation}.  
Ribero, Selbst \& Barocas, and Mittelstadt et. al. discuss multiple approaches to XAI: 
%\remco{replace with ``XXX et al. have suggested that'' -- more defensible} 
(1) using approximate models, (2) allowing users to explore how various input correlates with different outputs, and (3) replacing an opaque "black-box" model with a simpler “white-box" model ~\cite{RiberoLIME2016, SelbstBarocasIntuitive2018, MittelstadtRussellExplain2019}. However, none of these approaches is sufficient for our use case -- helping analysts determine when, and whether, to trust machine translation.
%\remco{this is a good sentence. It's the first time that the goal of the project is clearly stated. This sentence, more formalized (in terms of the actions that an analyst would need to do, should be put into the paragraph above}. 

Using an approximate MT model would significantly compromise translation accuracy%(if that were not the case we would use the simpler model for translation in general)
. Seeing pairs of foreign-language and translated text would be meaningless to non-speakers of that foreign language. And we know of no ``white-box" method for machine translation. \remco{this paragraph is great! Defend the three categories of XAI and this is a powerful motivation for why your work is needed.}

%\remco{Move the ``in this paper...'' paragraph up to here. Flow-wise, the motivation has been established. Now give the reader the solution. The next paragraph on QE is part of the solution of ``in this paper.'' Defending it prematurely feels apologetic.}

In this paper, we present and evaluate VeriCAT, which stands for verification of computer-assisted translation. VeriCAT is designed to help those using machine translation assess the quality of translated text. In particular, we focus on text snippets that have been translated from Russian into English by the FairSeq model ~\cite{ott-etal-2019-fairseq}. %In particular, VeriCAT aims to help users determine whether a particular MT sentence is trustworthy. For our particular use case, we build VeriCAT such that it provides ``explanations" for text translated from Russian $\rightarrow$ English by the FairSeq model~\cite{ott-etal-2019-fairseq}. 
VeriCAT consists of a Quality Estimation (QE) model combined with an easy-to-use interface. The objective of QE is to train a machine learning model to predict a quality score for translated text that is similar to what a human would assign to that translation~\cite{mauvcec2019machine}. VeriCAT's QE model is a trained version of OpenKiwi’s predictor-estimator QE~\cite{Kim2017PredictorEstimatorUM}. VeriCAT's interface shows users a predicted quality score (some valus out of a possible 100) for each individual sentence of machine-translated text (Figure \ref{fig:p3_predicted_quality}). %The main objective of the system is to provide users with additional information that they can use to determine whether a particular MT sentence is trustworthy.  %Furthermore, we run an empirical user study to see how users respond to VeriCAT, and if their individual differences affect how they use the tool. 

VeriCAT is novel in that it uses the output of a Quality Estimation model to provide context to analysts. Typically, QE is used by developers of MT models for validation and model improvement. However, we believe predicted quality scores can benefit analysts as well, by helping them determine when and whether to trust a particular MT sentence. While many MT accuracy metrics (such as BLEU score \cite{papineni-etal-2002-bleu}) provide information about the accuracy of a MT model in general, QE serves as a metric for \textit{individual sentences}. %Our approach differs from typical approaches to XAI in that it uses the output of one opaque model (the QE model) to provide context for the output of another opaque model (the MT model).


%In addition to turning to XAI, we spoke with analysts to better understand their needs around MT text. VeriCAT is novel in that it uses QE scores as an ``explanation" for MT text. While many common MT accuracy metrics (such as BLEU score \cite{papineni-etal-2002-bleu}) provide information about the accuracy of a MT model in general, QE serves as a metric for \textit{individual sentences}.The objective of QE is to train a machine learning model to predict a quality score for translated text that is similar to what a human would assign to the translation~\cite{mauvcec2019machine}. QE is typically used by developers of MT models for validation and model improvement. However, we believe QE can benefit analysts as well, by providing important contextual information about when and whether to trust a particular MT sentence. Our approach differs from typical approaches to XAI in that it uses the output of one opaque model (the QE model) to provide context for the output of another opaque model (the MT model). %However, we believe a quality score is a useful and intuitive way to help analysts determine whether MT text should be trusted or not.  

%Because analysts are in high demand, 
We evaluate VeriCAT with a quantitative %crowdsourced 
user study, where users are asked to perform an analogous task to that which motivated the development of VeriCAT. Participants are shown a passage of text translated from Russian $\rightarrow$ to English via the FairSeq model~\cite{ott-etal-2019-fairseq}. Participants are informed that they will be asked to answer two comprehension questions based on the text and are given the opportunity to request a human translation of any (or none) of the sentences in the passage before seeing the comprehension questions. We advise participants to select the sentence with the poorest quality translation for re-translation by a human, and we score participants based on whether they actually choose the lowest quality translation, as measured by a Direct Assessment score. \andrea{add sentence explaining DA score}

Our study shows that participants who have access to VeriCAT's predicted quality scores more frequently select the lowest quality translation (i.e. they perform better on the user study task), compared to participants who do not have VeriCAT's estimated quality scores. Moreover, we find correlations between participants' familiarity with MT tools and their self-rated expertise in AI and MT and their performance on the task. 

%Using QE as an explanation for MT text would be particularly useful in situations such as the Facebook scenario described above, where fluency of a MT is not an appropriate proxy measure for quality. However, there are drawbacks to using quality estimation as an explanation for MT. Namely, that QE models are also imperfect and there is no research showing whether users will heed quality scores over their own intuition. 

%Andrea stopped editing here ... on second read through as well

%\remco{need another paragraph or two to get into the meat of what VeriCAT is and/or what the user study is about. As a reader, at this point I'm still not 100\% sure what the contribution of the paper is... (1) is it VeriCat as a system? If so, it feels a little weird in that there's no declaration that VeriCat solves the problem posed in the motivation paragraph. (2) Related, if the contribution is VeriCat, then it's hard to believe that there are no other QE algorithms / software for MT. I feel like I have seen confidence bars before with a translation? Regardless, a quick google scholar search finds a patent on ``Method and apparatus for automated measurement of quality for machine translation''. So I imagine that others or out there? If so, why VeriCat and not other existing algorithms/software? (if other algorithms exist, then the motivation of comparing VeriCat to LIME isn't quite right) (3) is it the idea of using QE? If so, VeriCat is a means to an end. Then you need to say more about how you test the value of QE. (4) Personally, I feel like the contribution ought to be an integrated system of VeriCat + visualization interface?}

In the following sections we describe the design of VeriCAT. We briefly explain the QE model behind the system, and evaluate the system through a quantitative user study. Finally, we provide lessons learned from the design and evaluation of VeriCAT. % as well as a proposal for future work that incorporates users study results to iterate on the VeriCAT system.
In summary, we contribute the following: 

\begin{enumerate}
    \item The VeriCAT system which uses QE to help users decide if and when to trust sentences of machine translated text.   
   % \item \remco{possibly claim an ``iterative deisgn'' process that result in the interface for VeriCat? This is a little more compelling than the ``lessons learned'' at the end as a contribution but serves a similar goal? (Also, if this is established as a contribution, then there is more ground to claim UXAI as a lessons learned?)}
    \item An evaluation of VeriCAT demonstrating that it significantly improves participants' performance in identifying poor quality machine translations, and that differences between participants may effect how much they benefit from VeriCAT.
    \item Design implications and lessons learned from the design and evaluation of VeriCAT. 
\end{enumerate}

\section{Related Work}

Despite the growing ubiquity of MT, there is little research either on communicating the output of MT to end users, or on how visualization may help with this communication.  
To set the stage for out work, we provide high-level summaries of existing work in MT QE and in visualization for MT.  

\subsection{Machine Translation Quality Estimation} 
The study of MT began in the 1950's, but it was not until recently that significant advances greatly improved MT accuracy and usability. Despite these advances, however, human translators still outperform MT in overall accuracy and in preserving the original meaning of translated text\cite{mauvcec2019machine}. As use of MT becomes more widespread, problems can arise when translations are inaccurate. One way to address this problem is by using a second machine learning model to predict the quality of the MT model's output.  

There are many ways to evaluate MT quality. Numerous automatic metrics aim to approximate human judgement including: BLEU, NIST, METEOR, and TER\cite{mauvcec2019machine}. Additionally, there are some human-in-the-loop automatic judgements. For instance, HTER (human-mediated translation error rate)  attempts to capture the number of post edits made to a MT by a human translator\cite{mauvcec2019machine} and human judgements of translation fluency and adequacy\cite{snover2009Fluency} can be captured via a Direct Assessment (DA) score. While certain metrics are only appropriate for document-level quality assessment, others can be used to train a QE model to predict the quality of specific sentences. HTER and DA have previously been used to train sentence-level QE models~\cite{turchi2014adaptive, graham_baldwin_moffat_zobel_2017}.            

While many advancements have been made in MT QE, few efforts have studied how usable these tools are or how they might impact a user's ability to make decisions based on perceived translation quality. OpenKiwi \cite{UnBabel} performed a demo of a user interface for QE at ACL 2019; however, the team has not released their code, a demo, or a user study. There are also few tools designed to make QE accessible to non-MT experts. Avramidis created a GUI for this purpose; however, it still requires that users are proficient in Python and the command line\cite{avramidis2017QE}.    

At the IUI conference in 2020 researchers presented a demonstration titled: \textit{XAIT: An Interactive Website for Explainable AI for Text}. We reviewed the papers cited by this work and found that none of them related specifically to MT usability \cite{oduor2020XAIT}. In fact, the only usability study we found for MT was one by Martindale and Carpuat, which investigates how revealing MT errors in fluency and adequacy might change users' trust in MT. This work finds that poor fluency in translations can significantly influence users' trust of MT, but that trust is easily rebuilt\cite{martindaleFluency2018}.       

\subsection{Visualization and Machine Translation}

Though MT is becoming more widespread, there are few tools designed to help users understand its reliability. At the time of writing this paper we are aware of only one tool besides ours built to meet this need: Collins et al.'s lattice visualization, which illustrates uncertainty in MT text. While Collins et al. demonstrate that this method is effective for an instant messaging scenario~\cite{collins2007lattices}, we doubt its ability to scale to a use case involving full passages of text.  

Albrecht et al.'s human-AI collaborative system uses visualization to help users gain intuition about a translation's source language so that they may correct errors in MT text~\cite{albrecht2009chinese}. And DeNeefe et al. developed an interactive translation visualization tool called a DerivTool, which is intended to give users intuition about the MT model itself~\cite{deneefe2005interactively}. In contrast to these approaches, which emphasize source language or the underlying MT model itself, VeriCAT is distinct in that it provides contextual information about when and whether users should trust the output of MT, i.e. a particular snippet of MT text. 

%An area in which there is particular overlap between visualization and MT is the Digital Humanities. In this setting, the goal of visualization and MT is typically to enable comparisons of one text in several languages~\cite{janicke2017visual}. For example, ShakerVis is a tool for comparing different translations of \textit{Othello}~\cite{geng2015shakervis}. This work is similar to our in that the goal is making MT text more usable to users, however it differs significantly in  

%Our work differs from  

\section{Motivation}
\label{sec:design_requirements}

%Machine translation is becoming more widespread for a variety of uses. 
Today, finding machine translated text can be as simple as a Google search. However, as revealed by our review of the literature, there are limited tools that communicate to users when, and whether, they should trust MT text.  

%It is not uncommon for analysts to use MT text to scan text in languages they do not speak for potential threats. This is because although human translators outperform MT models, they are in high demand and expensive~\cite{mauvcec2019machine}. However, when analysts do not speak the original language of the text they are asked to analyze, they become %completely \remco{strong word -- hard to defend} 
%dependant on MT. Even if an analyst is aware that MT is not perfect, 

If a user of MT text is not a speaker of the source language and does not have additional contextual information, then their only basis for judging translation quality is fluency. Fluency refers to how well the text follows the target language's norms, taking into account grammar and clarity~\cite{mauvcec2019machine}.  

In many cases, fluency is a reasonable proxy for translation quality. However, it can also fail and the ``attack them" incident is a clear example. As a phrase, ``attack them" sounds fluent despite its very low quality translation. Another example is one we include in our user study. The Russian $\rightarrow$ English FairSeq model we use in this work is inaccurate when translating text in all caps. The model translates all caps Russian text that should read ``(BUT THEY DIDN'T HEAR IT)" to ``(BUT THIS HAPPENED)." While fluent, this translation is low quality because it obscures the meaning of the original text. We design VeriCAT to help users overcome these situations, where fluency is not an adequate proxy for translation quality.

%Similar to Collins et al.~\cite{collins2007lattices}, 
%Our goal was to design a system that provides contextual information for \textit{individual snippets} of translated text, as opposed to the quality of the MT model as a whole. Thus, VeriCAT uses a QE model to provide users with sentence-level estimates of MT quality.

%\begin{compacthang}
%\item \textbf{DR1:} The tool should support Russian text translated into English \remco{each of these should be elaborated} \remco{Note that for this particular DR, if you had mentioned the current tools used by the analysts, this DR might not be relevant because this is a given?}. 

%\item \textbf{DR:} The explanations should help users asses the quality of the translated text and quickly determine when a translation should to be verified by a human translator. 

%\item \textbf{DR3:} The explanations should be accessible to domain experts without expertise in AI or MT. 

%\item \textbf{DR4:} The tool should require little-to-no training to use. 

%\end{compacthang}

  

\begin{figure}
    \centering
    
    \begin{subfigure}[t]{0.45\textwidth}
        \centering
        \includegraphics[width=\linewidth]{p1_v_0.png} 
        \caption{Passage 1 (passage type 1), Human Quality condition} \label{fig:p1_human_quality}
    \end{subfigure}
    \hfill
     \begin{subfigure}[t]{0.45\textwidth}
        \centering
        \includegraphics[width=\linewidth]{p3_p_0.png} 
        \caption{Passage 3 (passage type 2), Predicted Quality condition} \label{fig:p3_predicted_quality}
    \end{subfigure}
    
    \caption{Select examples of stimuli. All stimuli are available in supplemental materials.}
    \label{fig:exp_stim}
    
\end{figure}

\section{VeriCAT System Overview}
VeriCAT is designed to help analysts quickly and efficiently identify Russian $\rightarrow$ English MT sentences with poor translation quality, particularly in cases where fluency is a poor proxy for translation quality. %VeriCAT predicts quality scores for Russian $\rightarrow$ English translations generated by the pretrained FairSeq model~\cite{ott-etal-2019-fairseq} that performed best at WMT 2019 (news task, the most recent results from the annual benchmark for MT).
%VeriCAT is novel in that it uses QE as a means of communicating to the user whether a specific sentence of translated text is trustworthy. 
Commonly used MT accuracy metrics such as BLEU score~\cite{papineni-etal-2002-bleu} provide information about the accuracy of a MT model in general, (for example, FairSeq has a BLEU score of 40.0 on Russian to English translation, calculated with the SacreBLEU standard \cite{post-2018-call}) but, unlike VeriCAT, they do not provide feedback on individual translated sentences.  
%Our goal was to design a system that provides contextual information for \textit{individual snippets} of translated text, as opposed to the quality of the MT model as a whole.    
Below, we describe the training dataset for VeriCAT's QE model, the QE model itself, and VeriCAT's user interface.    

\subsection{Training Dataset}
The VeriCAT QE model is finetuned on a dataset composed of 7,000 labeled sentence pairs. The source of this text is passages from Reddit or Russian Proverbs from wikiquotes. The training dataset is curated from these sources because they represent types of text on which machine translation models are challenged. Each sentence is translated using the pretrained FairSeq model~\cite{ott-etal-2019-fairseq} that performed best at WMT 2019 (news task, the most recent results from the annual benchmark for MT). Each sentence has 3 Direct Assessment (DA) score quality judgments by human translators. These DA scores were labeled by ModelFront. Each DA score is rated on a scale from 1-100, with 100 representing a perfect translation. Across the dataset the average score is 68. These labeled data were also contributed to the World Machine Translation Workshop (Nov 2020) as part of the Quality Estimation Shared Task\footnote{statmt.org/wmt20}.  


\subsection{Quality Estimation Model}
Quality Estimation benchmarks are set annually at the World Machine Translation (WMT) QE Shared task. At the time of this study, the most accurate QE model available in the open source is the Predictor-Estimator model~\cite{Kim2017PredictorEstimatorUM}, open-sourced by OpenKiwi\footnote{https://github.com/Unbabel/OpenKiwi/tree/master/kiwi}~\cite{UnBabel} and the benchmark for WMT 2020. We pretrained the predictor model on the same parallel datasets the FairSeq translation model~\cite{ott-etal-2019-fairseq} was trained on. We finetuned the estimator model on the novel Russian-English QE dataset detailed above, tuning the following hyperparameters from the baseline model: epochs, hidden LSTM layers, learning rate, batch size, and dropout. We obtained a Pearson correlation of 0.62 on the development set, which we used to test since the shared task test set was not known to us. We ran inference with this model to generate the predictions for the VeriCAT UI, and confirmed the correlation between predicted and actual scores for this data subset was 0.67, in line with the model's expected performance. 


\subsection{User Interface}

The VeriCAT user interface is designed to provide context to help users assess the trustworthiness of a Russian $\rightarrow$ English translation via the FairSeq model.
Following Tufte's advice of \textit{above all else show the data}, the VeriCAT interface is designed simply, with a high data-ink ratio~\cite{tufte}.  
In the VeriCAT user interface, a passage of text is broken down into individual sentences. For each sentence, users see the original (Russian) text, the FairSeq translation (English), and VeriCAT’s quality score for that sentence (Figure \ref{fig:p3_predicted_quality}). Quality scores are represented with a simple horizontal bar, where the percentage of the bar that is colored represents the score on a scale from 1 to 100. For clarity, the numerical value of the quality score is also displayed. Unlike the uncertainty lattice developed by Collins et al.~\cite{collins2007lattices}, which drills down to word-level uncertainty, our approach focuses on sentences as a whole. These sentence-level quality scores are intended to help users quickly assess the translation quality for each sentence, to determine if it needs further inspection by a human. 


%\ab{I think we might want to move this next paragraph to future work instead of here. I'm just thinking since we don't test this version or provide a use case for it it might feel a little out of flow for new readers. Alternatively I think we could talk about this version first and say we came to the version we test through iterative design.}

%In addition to the version described above, we created a training version of VeriCAT.
%With this version, when viewing the training dataset, users have the option to switch to “God-mode” which allows them to see the hand-corrected version of each translated sentence — a “ground truth” for the dataset. In this mode, the interface also shows world-level errors in the translated text. Incorrect words are displayed in red, missing words are indicated with an underscore, and deletions are shown in red text with a strikethrough. Types of errors per sentence are aggregated and summarized below the Quality Score for each sentence. %An example of this version of VeriCAT is shown in Figure \ref{fig:teaser}.  


%https://github.com/Lab41/VeriCAT-UI

  



\section{Quantitative Evaluation} 

In this section we describe an experiment we ran to evaluate whether VeriCAT improves users' ability to identify when Russian $\rightarrow$ English translated text is of sufficiently low quality to be untrustworthy. The following sections outline the study in more detail. 

%\subsection{Explanation Techniques}
\subsection{Experimental Conditions}

We perform a between-subjects 3 condition experiment. In the baseline condition, participants see a passage of original Russian text broken into sentences, along with the corresponding MT. No additional information is provided to the participant. Essentially, the baseline condition is the VeriCAT interface without quality scores. We chose this as our baseline (as opposed to raw input and output to the FairSeq translation model) to avoid confounding the benefits of VeriCAT with the benefits of clean text formatting. % keep the baseline condition as consistent with other conditions as possible.    

In addition to the baseline condition, we include two experimental conditions, each of which incorporates a version of VeriCAT. 
In the first experimental condition, participants use the VeriCAT system with quality scores output by its QE model (Figure \ref{fig:p3_predicted_quality}) to perform the experimental task. 
In the second experimental condition, participants use the VeriCAT interface, but instead of showing predicted quality scores, the interface shows human-generated direct assessment (DA) scores of the translated text (Figure \ref{fig:p1_human_quality}). 
We refer to the experimental conditions as Predicted Quality (for VeriCAT with its QE model scores), and Human Quality (for the VeriCAT interface, with DA scores). 

In the event that VeriCAT fails to improve participants' performance, we want to design the experiment such that we can distinguish between failure due to the system design vs. failure due to variance in predicted quality scores. In other words, we include the Human Quality condition so that we can rule out variance in predicted quality scores as a mode of failure. 

\subsection{Passage Type}
We postulate that without quality scores, users will rely on fluency as a proxy for translation quality. Prior work shows that this is the case when it comes to trust of MT \cite{martindaleFluency2018}. However, there are cases where this technique fails, as described in Section \ref{sec:design_requirements}.  
In our experiment, each translated sentence falls into one of four categories: (1) \textit{poor fluency + poor quality}, (2) \textit{poor fluency + good quality}, (3) \textit{good fluency + poor quality}, (4) \textit{good fluency + good quality}. We expect sentences in the \textit{good fluency + poor quality}, and \textit{poor fluency + good quality} categories to be the most difficult for participants to assess. To balance the experiment with these different sentence types we arrange two different types of passages described below: 

\begin{compacthang}
    \item \textbf{Type 1} \textit{Good fluency + good quality, good fluency + poor quality, poor fluency + good quality}. Passage 1 is of this type and is shown in Figure \ref{fig:p1_human_quality}.    
    \item \textbf{Type 2} \textit{Good fluency + good quality, poor fluency + good quality, poor fluency + poor quality}. Passage 3 is of this type and is shown in Figure \ref{fig:p3_predicted_quality}.     
\end{compacthang}

Both of these passage types are designed to test if participants will heed quality scores or (their own intuition based on) fluency, as a primary indicator of poor quality translation. 

\subsection{Task} 
We run a 3 condition experiment: Baseline, Human Quality, and Predicted Quality. Within each condition we include 2 passages of type 1 followed by 2 of type 2. The experimental task replicates a user determining if a MT needs a human translation as closely as possible within a controlled setting. 

\begin{table}[h!]
\resizebox{0.50\textwidth}{!}{%
\begin{tabular}{c|c|c}
\hline
\begin{tabular}[c]{@{}c@{}}Human Quality Score \\ (Human Quality)\end{tabular} & \begin{tabular}[c]{@{}c@{}}Predicted Quality Score \\ (Predicted Quality)\end{tabular} &  \begin{tabular}[c]{@{}c@{}}Baseline\\ (Baseline)\end{tabular} \\ \hline
                                                                    59                                                                       & 59                                                                                         & 47                                                        \\ \hline
\end{tabular}
}
\caption{N for each condition. }
\label{tab:exp_N}
\end{table}

Participants are shown four passages of text. Within each passage, text is broken into three sentences. For each sentence, we show the original Russian text and the FairSeq English translation. Participants in either of the quality score conditions also see a quality score for each sentence (Figure \ref{fig:exp_stim}). Participants are instructed to decide if any of the sentences in the passage should be re-translated by a human and are given the chance to select sentence 1, 2, or 3 of the passage for re-translation by a human, or to opt for no-retranslation. After this selection, the original passage and machine translation remain on the screen and the human translation of whichever sentence they selected is shown. At this point, participants are asked to answer two information retrieval questions. 
 %The prompt before each passage is as follows:

%\begin{quote}
%Below is a passage written in Russian with an English translation generated by artificial intelligence. When you are ready, you will be asked to answer two comprehension questions. \textbf{You may select up to one section of the passage for re-translation by a human.} We recommend selecting the section you judge to have the poorest machine translation for re-translation by a human. 

%Click ``I’m ready to see the questions'' when you are ready to see the comprehension questions. \textbf{If you would like a human re-translation of any section of the passage you must request it BEFORE you click ``I’m ready to see the questions''.} Once you click ``I’m ready to see the questions'', comprehension questions will appear on the page alongside the Russian test, English Machine translation, and any human translations you requested.  
%\end{quote}


\subsection{Participants} 

We recruited 193 participants from Amazon Mechanical Turk. Participation was restricted to workers in the United States with an approval rating of greater than 90 percent who do not speak Russian or Ukrainian. Participants were paid a base rate of USD $1.60$ for participation. Before analysis, participants who answered information retrieval questions for passage 1 and passage 2 (attention check questions) incorrectly were dropped from analysis ($N = 28$).  This left $N = 165$ participants distributed among stimuli as shown in Table \ref{tab:exp_N}. Demographics of participants are shown in Table \ref{tab:exp_demo}.


\begin{table}[h!]
\begin{threeparttable}[b]
\begin{tabular}{ll}
\hline
N                                                                                & 165                                                                                                                                                    \\ \hline
Age                                                                              & \begin{tabular}[c]{@{}l@{}}18-24: 6.7\%, 25-34: 45.5\%, 35-44: 24.8\%, \\ 45-54: 14.5\%, 55-64: 7.3\%, 65+: 1.2\%\end{tabular}                                     \\ \hline
Gender                                                                           & \begin{tabular}[c]{@{}l@{}}Female: 38.2\%, Male: 61.2\%, Non-binary: 0.6\%\end{tabular}                                                             \\ \hline
Education                                                                        & \begin{tabular}[c]{@{}l@{}}High School: 13.3\%, Associates: 12.1\%, Bachelors: 60.0\%, \\ Masters: 11.5\%, Professional: 2.4\%, Doctorate: 0.6\%\end{tabular}                            \\ \hline
\end{tabular}
\end{threeparttable}
\caption{Participant demographics.}
\label{tab:exp_demo}
\end{table}

\subsection{Procedure}

The experiment follows an approved protocol per \textit{redacted for anonymity}’s company policy and was posted as a HIT on Amazon Mechanical Turk. Workers who accept the HIT follow a link to the experiment. After providing informed consent, participants see an instruction page explaining the experiment. This page explains that they will see 4 passages of text translated from Russian to English by AI and that they will be asked to answer two comprehension questions for each passage. The instructions also explain that before seeing the comprehension questions the participant will have the opportunity to select one of the sentences in the passage for re-translation by a human. After the instruction page, participants see the four passages one at a time. They can take as much time as they need before clicking a button to select which section of the passage they want to re-translate and viewing the passage questions. After completing the main task, participants complete a short post-experiment questionnaire, the Tolerance for Ambiguity Survey from Geller et al. 1993 \cite{gellerTolerance1993}, a short demographic questionnaire, and have the option to provide any additional feedback.

\subsection{Hypotheses}

%We postulate that providing quality estimations for each sentence will help participants identify the sentence of lowest quality for re-translation only in cases where there is a significant difference between scores. In other words, we expect participants in the Human Quality condition to perform significantly better than participants in the Predicted Quality condition. 

%We evaluate VeriCAT with DA vs. QE scores, and with respect t our design requirements (Section \ref{sec:design_requirements}). To gather a more nuanced understanding of how individual users may or may not benefit from our approach to explaining MT we also test a number of hypotheses based on individual user differences. 

We analyse the results of this experiment according to the following hypotheses: 


\begin{compacthang}
    \item \textbf{H1}: Participants in the VeriCAT (with quality scores) conditions will have higher accuracy in identifying which sentence in a passage is of low quality and should be re-translated than participants in the baseline condition. 
    \item \textbf{H2}: Participants in the VeriCAT (with quality scores) conditions will have a greater change in trust of machine translation than participants in the baseline condition. 
    \item \textbf{H3}: Participants’ tolerance for ambiguity will correlate with how well they are able to use VeriCAT (with quality scores) to perform the experimental task.
    \item \textbf{H4}: Participants’ experience using machine translation will correlate with how well they are able to use VeriCAT (with quality scores) to perform the experimental task.
    \item \textbf{H5}: Participants’ self-rated expertise in AI, MT, visualization, and statistics will correlate with how well they are able to use VeriCAT (with quality scores) to perform the experimental task.   
\end{compacthang}

\begin{figure}*
    \centering
    
    \cbox{bar-noXai} \textit{Baseline} \quad
    \cbox{bar-Qual} \textit{Human Quality} \quad
    \cbox{bar-PredictQ} \textit{Predicted Quality} \quad
    
    \begin{subfigure}[t]{0.45\textwidth}
        \centering
        \scalebox{0.7}{
        \begin{bchart}[step=.25,max=1,width=\linewidth]
        \bcbar[color=bar-noXai]{.06}
        \bclabel{\textit{Good Fluency}}
        \bcbar[color=bar-Qual]{.51}
        \bclabel{\textit{Poor score}}
        \bcbar[color=bar-PredictQ]{.37}
        \bcskip{6pt}
        
        \bcbar[color=bar-noXai]{.15}
        \bclabel{\textit{Good Fluency}}
        \bcbar[color=bar-Qual]{.15}
        \bclabel{\textit{Good score}}
        \bcbar[color=bar-PredictQ]{.07}
        \bcskip{6pt}
        
        \bcbar[color=bar-noXai]{.38}
        \bclabel{\textit{Poor Fluency}}
        \bcbar[color=bar-Qual]{.14}
        \bclabel{\textit{Good score}}
        \bcbar[color=bar-PredictQ]{.31}
        \bcskip{6pt}
        
        \bcbar[color=bar-noXai]{.40}
        \bclabel{\textit{No}}
        \bcbar[color=bar-Qual]{.20}
        \bclabel{\textit{Re-translation}}
        \bcbar[color=bar-PredictQ]{.25}
        
        \bcxlabel{Proportion of Participants Selecting}
        \end{bchart}}
        \caption{Passage 1} 
        \label{fig:exp_p1_prop_answers}
    \end{subfigure}
    \hfill
    \begin{subfigure}[t]{0.45\textwidth}
        \centering
        \scalebox{0.7}{
        \begin{bchart}[step=.25,max=1,width=\linewidth]
        \bcbar[color=bar-noXai]{.06}
        \bclabel{\textit{Good Fluency}}
        \bcbar[color=bar-Qual]{.53}
        \bclabel{\textit{Poor score}}
        \bcbar[color=bar-PredictQ]{.37}
        \bcskip{6pt}
        
        \bcbar[color=bar-noXai]{.13}
        \bclabel{\textit{Good Fluency}}
        \bcbar[color=bar-Qual]{.12}
        \bclabel{\textit{Good score}}
        \bcbar[color=bar-PredictQ]{.08}
        \bcskip{6pt}
        
        \bcbar[color=bar-noXai]{.21}
        \bclabel{\textit{Poor Fluency}}
        \bcbar[color=bar-Qual]{.15}
        \bclabel{\textit{Good score}}
        \bcbar[color=bar-PredictQ]{.20}
        \bcskip{6pt}
        
        \bcbar[color=bar-noXai]{.60}
        \bclabel{\textit{No}}
        \bcbar[color=bar-Qual]{.20}
        \bclabel{\textit{Re-translation}}
        \bcbar[color=bar-PredictQ]{.34}
        
        \bcxlabel{Proportion of Participants Selecting}
        \end{bchart}}
        \caption{Passage 2} 
        \label{fig:exp_p2_prop_answers}
    \end{subfigure}
    \hfill
     \begin{subfigure}[t]{0.45\textwidth}
        \centering
        \scalebox{0.7}{
        \begin{bchart}[step=.25,max=1,width=\linewidth]
        \bcbar[color=bar-noXai]{.26}
        \bclabel{\textit{Poor Fluency}}
        \bcbar[color=bar-Qual]{.69}
        \bclabel{\textit{Poor score}}
        \bcbar[color=bar-PredictQ]{.68}
        \bcskip{6pt}
        
        \bcbar[color=bar-noXai]{.19}
        \bclabel{\textit{Poor Fluency}}
        \bcbar[color=bar-Qual]{.05}
        \bclabel{\textit{Good score}}
        \bcbar[color=bar-PredictQ]{.15}
        \bcskip{6pt}
        
        \bcbar[color=bar-noXai]{.09}
        \bclabel{\textit{Good Fluency}}
        \bcbar[color=bar-Qual]{.10}
        \bclabel{\textit{Good score}}
        \bcbar[color=bar-PredictQ]{.03}
        \bcskip{6pt}
        
        \bcbar[color=bar-noXai]{.47}
        \bclabel{\textit{No}}
        \bcbar[color=bar-Qual]{.15}
        \bclabel{\textit{Re-translation}}
        \bcbar[color=bar-PredictQ]{.14}
        
        \bcxlabel{Proportion of Participants Selecting}
        \end{bchart}}
        \caption{Passage 3} 
        \label{fig:exp_p3_prop_answers}
    \end{subfigure}
    \hfill
    \begin{subfigure}[t]{0.45\textwidth}
        \centering
        \scalebox{0.7}{
        \begin{bchart}[step=.25,max=1,width=\linewidth]
        \bcbar[color=bar-noXai]{.45}
        \bclabel{\textit{Poor Fluency}}
        \bcbar[color=bar-Qual]{.64}
        \bclabel{\textit{Poor score}}
        \bcbar[color=bar-PredictQ]{.64}
        \bcskip{6pt}
        
        \bcbar[color=bar-noXai]{.02}
        \bclabel{\textit{Poor Fluency}}
        \bcbar[color=bar-Qual]{.12}
        \bclabel{\textit{Good score}}
        \bcbar[color=bar-PredictQ]{.08}
        \bcskip{6pt}
        
        \bcbar[color=bar-noXai]{.11}
        \bclabel{\textit{Good Fluency}}
        \bcbar[color=bar-Qual]{.10}
        \bclabel{\textit{Good score}}
        \bcbar[color=bar-PredictQ]{.12}
        \bcskip{6pt}
        
        \bcbar[color=bar-noXai]{.43}
        \bclabel{\textit{No}}
        \bcbar[color=bar-Qual]{.14}
        \bclabel{\textit{Re-translation}}
        \bcbar[color=bar-PredictQ]{.15}
        
        \bcxlabel{Proportion of Participants Selecting}
        \end{bchart}}
        \caption{Passage 4} 
        \label{fig:exp_p4_prop_answers}
    \end{subfigure}
    
    \caption{Proportions of participants selecting each type of sentence for re-translation by passage.}
    \label{fig:exp_prop_answers}

\end{figure}

\subsection{Findings}

We consider participants' answers correct if they select the sentence with the lowest quality score to be re-translated. In all cases, the sentence with the lowest score in the predicted quality score condition is the same as the sentence with the lowest score in the human quality score condition. To calculate overall user performance accuracy, we sum the number of correct answers across all four passages and divide by 4. The following analyses use this outcome measure to test the hypotheses listed above. Analysis scripts and de-identified data are available in supplemental materials. 

\subsubsection{\textbf{Does VeriCAT help?}}

We start by looking at which sentences people chose to re-translate for each condition and passage. For all passages, we see participants in the quality score conditions are, on average, most accurate at selecting the correct sentence for re-translation. Interestingly, we see that participants in the baseline condition often opt for no-retranslation. Proportions of participants giving each answer for each passage and condition are shown in Figure \ref{fig:exp_prop_answers}.

To test if the differences we observe are significant we run a Kruskal-Wallis test of $overall\_score \sim condition$ \footnote{We use a Kruskal-Wallis test because according to the Shapiro-Wilk Normality test $overall\_score$ is not normally distributed ($W = 0.87, p < 0.001$).} and find a significant difference across conditions ($H(2) = 29.7, p < 0.001$). A post-hoc Dunn’s multiple comparisons test with a Bonferroni corrected alpha ($0.02$) shows significant pairwise differences between Human Quality and baseline ($Z = 5.2, p < 0.01$), and baseline and Predicted Quality ($Z = -4.3, p < 0.01$). Figure \ref{fig:exp_overall_distribution} shows $overall\_score$ distributions by condition. 

\begin{figure}[h!]
  \centering
  \begin{minipage}[b]{0.45\textwidth}
    
   %\begin{figure}[h!]
    %\centering
    \includegraphics[width=0.95\textwidth]{exp2_overall_distribution.png}
    \caption{Boxplot of overall score by condition.}
    \label{fig:exp_overall_distribution}
    %\end{figure}
    
  \end{minipage}
  \hfill
  \begin{minipage}[b]{0.45\textwidth}
  
    %\begin{figure}[h!]
    %\centering
    \includegraphics[width=0.95\textwidth]{exp2_delta_trust.png}
    \caption{Boxplot of change in trust by condition.}
    \label{fig:exp_delta_trust}
    %\end{figure}
    
  \end{minipage}
\end{figure}

These results suggest that VeriCAT's sentence-level quality scores can significantly improve participants’ performance over baseline, regardless of whether the scores come from (noisy) predictions generated by the QE model, or human judgments (DA scores). Therefore, we \textbf{accept H1}. The results also do not indicate a significant difference in performance between predicted quality scores and human quality scores. We interpret this to suggest that, though imperfect, VeriCAT's predicted quality scores are comparable to the human quality scores in their impact on users' ability to perform the task. 

\subsubsection{\textbf{Does VeriCAT affect trust?}}

Before the experimental task, we ask participants: “Please rate how much you trust artificial intelligence to correctly translate sentences from a language you do not speak into a language you do speak from 1 (No trust) to 5 (Complete trust).”  We ask the same question again at the end of the experiment. To analyze whether different conditions inspire a change in participants' reported trust, we calculate $delta\_trust$ for each participant by subtracting their answer to the pre-experimental task trust question from their answer to the post-experimental task trust question. 

We run a Kruskal-Wallis test of $delta\_trust \sim condition$ \footnote{We use a Kruskal-Wallis test because according to the Shapiro-Wilk Normality test $delta\_trust$ is not normally distributed ($W = 0.74, p < 0.001$).} and find no significant difference across conditions ($H(2) = 3.5, p = 0.17$). This suggests that neither Human Quality or Predicted Quality scores significantly affect participants’ trust in machine translation, thus we \textbf{reject H2}. Average change in trust by condition is shown in Figure \ref{fig:exp_delta_trust}.

\subsubsection{\textbf{Is the efficacy of VeriCAT influenced by participants' individual differences?}}

Prior work indicates that individual user differences can play a strong role in how well users utilize a visualization for a problem solving task\cite{liuSurvey2020}. However, there is little research investigating how individual differences can affect understanding, trust, and use of AI. In an effort to investigate this question, we capture three different individual differences of participants (intolerance for ambiguity, usage of MT, and self-rated expertise) and analyse whether there are any correlations between these measures and participants’ $overall\_score$. Our findings follow.

\begin{figure}[h!]
  \centering
  
    \cbox{bar-noXai} \textit{Baseline} \quad
    \cbox{bar-Qual} \textit{Human Quality} \quad
    \cbox{bar-PredictQ} \textit{Predicted Quality} \quad
    
  \begin{minipage}[b]{0.45\textwidth}
    
   %\begin{figure}[h!]
    %\centering
    \includegraphics[width=0.95\textwidth]{exp2_intol_ambiguity.png}
    \caption{Regression lines of overall score and intolerance for ambiguity by condition.}
    \label{fig:exp_intol_ambiguity}
    %\end{figure}
    
  \end{minipage}
  \hfill
  \begin{minipage}[b]{0.45\textwidth}
  
    %\begin{figure}[h!]
    %\centering
    \includegraphics[width=0.95\textwidth]{images/exp2_MT_useage.png}
    \caption{Regression lines of overall score and frequency of MT usage by condition.}
    \label{fig:exp_MT_use}
    %\end{figure}
    
  \end{minipage}
\end{figure}

\paragraph{\textbf{Does intolerance of ambiguity affect performance?}} 

Given the survey instrument we use to measure Intolerance for Ambiguity\cite{gellerTolerance1993} participants’ scores could range from 7 (extremely low intolerance for ambiguity) to 49 (extremely high intolerance for ambiguity). In our study, the median score for participants' intolerance for ambiguity is 30. 

To determine if there could be a linear relationship between participants' $overall_score$ and $intolerance\_for\_ambiguity$, we test for a significant Pearson correlation between the two within each condition. Regression lines for each condition are shown in Figure \ref{fig:exp_intol_ambiguity}. We find no significant correlation in any condition (Human Quality -- ($r(57) = -0.14, p = 0.29$), Predicted Quality -- ($r(57) = 0.03, p = 0.81$), baseline -- ($r(45) = -0.01, p = 0.95)$), and thus do not perform any linear regressions. This suggests that intolerance for ambiguity does not have a significant effect on participants' performance, thus we \textbf{reject H3}. 

\paragraph{\textbf{Does regular usage of machine translation affect performance?}}

We ask participants to rank how often they use Google Translate and Facebook Translate on a five-point scale of Never, Yearly, Monthly, Weekly, Daily. We assign weights to each point on the scale ranging from 1 (Never) to 5 (Daily) and use these to calculate a machine translation usage score for each participant. The higher this score, the more often a participant indicates using MT. 

To determine if there could be a linear relationship between participants' $overall_score$ and $MT_usage$, we test for a significant Pearson correlation between the two within each condition.
Regression lines for each condition are shown in Figure \ref{fig:exp_MT_use}. We find a significant correlation between frequency of MT usage and overall score for participants in the Human Quality condition ($r(57) = -0.32, p < 0.05$), and no significant correlation in any other condition (Predicted Quality -- ($r(57) = -0.21, p = 0.11$), baseline -- ($r(45) = -0.23, p = 0.11$)). We run a linear regression for the Human quality condition and find a significant effect between $overall_score$ and $MT_usage$ ($F(1, 57) = 6.4, p < .05, R^2 = 0.10$), with $MT_usage$ as a significant predictor ($t = -2.54, p < 0.05$). This suggests that in the Human Quality condition, as participants’ frequency of MT usage increases, their performance decreases, thus we \textbf{partially accept H4}.

\begin{figure}[h!]
    \centering
    
    \cbox{bar-noXai} \textit{Baseline} \quad
    \cbox{bar-Qual} \textit{Human Quality} \quad
    \cbox{bar-PredictQ} \textit{Predicted Quality} \quad
    
    \includegraphics[width=0.50\textwidth]{exp2_expert.png}
    \caption{Regression lines of overall score and self-rated expertise by condition.}
    \label{fig:exp_expert}
\end{figure}

\begin{table}[h!]
\resizebox{0.90\textwidth}{!}{%
\begin{tabular}{llcc}
\hline
\multicolumn{1}{c}{Measure}                            & \multicolumn{1}{c}{Condition} & Result & Regression    \\ \hline
\multirow{4}{*}{Self-rated expertise in AI}            & Human Quality              & $\mathbf{r(57) = -0.27, p < 0.05}$ & 
$\mathbf{F(1, 57) = 4.39, p < 0.05, R^2 = 0.07; t = -2.10, p < 0.05}$  \\
                                                       & Predicted Quality          & $\mathbf{r(57) = -0.29, p < 0.05}$ & 
$\mathbf{F(1, 57) = 5.24, p < 0.05, R^2 = 0.08; t = -2.29, p < 0.05}$         \\
                                                       & Baseline                     & $r(45) = 0.02, p = 0.91$                     \\ \hline
\multirow{4}{*}{Self-rated expertise in MT}            & Human Quality              & $\mathbf{r(57) = -0.31, p < 0.05}$ & 
$\mathbf{F(1, 57) = 6.26, p < 0.05, R^2 = 0.10; t = -2.50, p < 0.05}$ \\
                                                       & Predicted Quality          & $\mathbf{r(57) = -0.40, p < 0.01}$ & 
$\mathbf{F(1, 57) = 10.55, p < 0.01, R^2 = 0.16; t = -3.25, p < 0.01}$          \\
                                                       & Baseline                     & $r(45) = -0.13, p = 0.39$                    \\ \hline
\multirow{4}{*}{Self-rated expertise in visualization} & Human Quality              & $r(57) = -0.24, p = 0.06$                    \\
                                                       & Predicted Quality          & $\mathbf{r(57) = -0.31, p < 0.05}$ & 
$\mathbf{F(1, 57) = 6.18, p < 0.05, R^2 = 0.10; t = -2.49, p < 0.05}$                  \\
                                                       & Baseline                     & $r(45) = -0.11, p = 0.48$                    \\ \hline
\multirow{4}{*}{Self-rated expertise in statistics}    & Human Quality              & $r(57) = -0.21, p = 0.10$                    \\
                                                       & Predicted Quality          & $\mathbf{r(57) = -0.36, p < 0.01}$ & 
$\mathbf{F(1, 57) = 8.75, p < 0.01, R^2 = 0.13; t = -2.96, p < 0.01}$          \\
                                                       & Baseline                     & $r(45) = -0.11, p = 0.46$                    \\ \hline
\end{tabular}%
}
\caption{Pearson correlation results for each self-rated expertise measure by condition. Significant results are in \textbf{bold} and regression results are included.}
\label{tab:exp1_expertise_stats}
\end{table}


\paragraph{\textbf{Does participants' self-rated expertise affect performance?}}

We ask participants to rate their own expertise from 1 (Novice) to 5 (Expert) in four areas: AI, MT, visualization, and statistics. 

To determine if there could be a linear relationship between participants' $overall_score$ and $expertise$, we test for a significant Pearson correlation between the two within each condition. Regression lines for each condition are shown in Figure \ref{fig:exp_expert}.
In the Human Quality, and Predicted Quality conditions we find significant correlations between self rated expertise in AI and MT and $overall\_score$. In addition, we find significant correlations between self-rated expertise in visualization and statistics and overall score in the Predicted Quality condition. In all of these cases, linear regression shows self-rated expertise is a significant predictor of $overall\_score$. 
(Analysis results for all of these tests are listed in Table \ref{tab:exp1_expertise_stats}.) Overall, our results suggest that in both quality score conditions, as self-rated expertise in AI and MT (and in the case of Predicted Quality, visualization and statistics) increase, $overall\_score$ decreases, thus we \textbf{accept H5}.


\subsection{\textbf{Qualitative Feedback}}

Qualitative feedback from our study indicates that users are satisfied with VeriCAT. We ask participants in the Predicted Quality condition if they would have liked to see any additional information and only 5 of the 59 ($8.5\%$) participants answered yes. We interpret this to indicate that VeriCAT provides adequate information for users to perform the experimental task.       

\subsection{Discussion} 

In running this experiment, we sought to test if VeriCAT's method of showing sentence-level quality scores improves users' performance in a task that asks them to identify poor quality machine translations. We find that both Human and Predicted quality scores significantly improve user performance compared to the baseline condition. In addition, we find that although user performance is slightly lower with Predicted Quality scores (generated by VeriCAT's QE model) compared to Human Quality scores (generated by DA of translations by humans), this difference is not significant. 

We do not find evidence that VeriCAT impacts how participants' respond to our questions regarding trust of MT. There are a few potential explanations for this. It may be that participants come into the experiment with prior skepticism of MT quality. Even when translation quality scores help them perform the experimental task more accurately, the scores may not necessarily change their perception of the trustworthiness of MT in general. It may also be that our experimental task is not realistic enough to force users to consider their trust of MT output. Future work should explore situations where the stakes for users are higher. For example, future studies might ask users to flag sentences as inappropriate based on MT, or whether they would re-post a passage based on the MT. Scenarios such as these may elicit a stronger evaluation of trust from participants.

Unlike participants in the Human Quality condition, we do not find a significant correlation between participants' overall score and frequency of MT use in the Predicted Quality condition. However, we do find a significant negative correlation between participants' self-rated expertise in AI and MT and overall score in the Human and Predicted Quality conditions.  

This finding is surprising, as we would expect people who are more familiar with MT through frequent usage to have a better understanding of its limitations. Similarly, we would expect people with more expertise in MT and AI to have a better understanding of MT limitations and of how to read and process quality indicators. We postulate that in both of these cases we are observing participants' overconfidence leading to errors. It may be that participants who use MT more often become more comfortable with it and, as a result, see less need to rely on quality scores to identify poor translations. It may also be that these participants are less invested in attempting to perform the task as directed. We postulate that the same is true of self-rated experts in MT and AI. I.e. that participants who think they are better at understanding the underlying mechanisms of MT may be more likely to rely on their own quality assessments instead of those shown by VeriCAT.

Our findings suggest that novices are more likely to follow quality score guidance, while those with more familiarity or expertise in the area may ignore the guidance of quality scores in favor of their own judgement. When designing interfaces, designers should pay specific attention to the differences between these populations and appreciate that their system design may need to accommodate different users in different ways. 

Overall, our findings indicate that VeriCAT provides contextual information about the quality of MT text that is useful and intuitive for users. In particular, we see that VeriCAT's quality scores can help users overcome situations in which fluency of translated text is a poor proxy for quality. Our hope is that future work will continue in this area to investigate additional methods to provide MT users helpful indications of reliability and trustworthiness.     




%\input{sections/qual_eval}
\section{Lessons Learned}

Through the process of conceptualizing, testing, and refining VeriCAT we show that QE can provide context to translations and help users to appropriately trust MT text.  
Below, we present several lessons learned from the evaluation of VeriCAT.     

\paragraph{\textbf{Test for behaviors and beliefs}} In this study we saw that adding quality scores to MT text did not significantly change participants' self-reported trust of MT. However, we did see a significant change in behavior when participants were exposed to quality scores. Moreover, we observe that participants tend to opt for no re-translation when presented with a passage of machine translated text and no additional information, which suggests they may believe re-translation is unnecessary, or may be unwilling to choose without sufficient information. In contrast, we observe participants in quality score conditions tend to opt for a human re-translation. This suggests a more appropriate level of trust in the MT output, as participants are choosing to double check portions of it. Based on this mismatch of self-reported trust and behavior, we recommend testing for changes in behavior as well as collecting self-reported measures when evaluating the performance of tools designed to inform users' trust.     

\paragraph{\textbf{Plan for overconfidence}} Interestingly, we found negative correlations between participants' usage of MT tools and overall scores and between participants' self-rated expertise and overall scores. Our results imply that it may be necessary to make indications of poor translation quality more alarming to users who use MT more often. Similarly, users who have high confidence in their AI and MT expertise may need intensified warnings of poor translation quality compared to users with low confidence in these areas. In short, our results suggest that there is likely not a ``one size fits all" solution to helping users have appropriate trust of MT text; in addition to customizing such tools on a macro-level for various domains they should be customized on a micro-level to account for differences in individual users.   

%\paragraph{\textbf{Employ a user-centered design process}} Perhaps the most important lesson learned in building and evaluating VeriCAT is that a user-centered design process is invaluable to XAI development. Through this experiment we were able to establish that sentence-level quality scores are a viable option for improving analysts' performance in identifying poor quality MT text. However, as stated earlier, there are multiple ways in which translation quality could be communicated to users, and we would strongly urge future work to look into other options. Further, we believe our study is an excellent example of how the outcome of usability testing for XAI can inform interface design, and also model development. For example, if we had found that DA scores significantly improved participants' performance but QE scores did not, before abandoning QE as an explanation option we would have iterated on the QE model to see if making it more accurate (i.e. closer to ground truth) could reduce the difference in performance between participants using VeriCAT with DA vs QE scores.     


%Revisions to interface (\textbf{interface version 2})

%Insights about model design (\textbf{model version 3})

\section{Conclusion}

Because human translators are limited, intelligence analysts can find themselves dependant on machine translations of text in order to do their jobs. However, MT is not perfect, and acting on miss-translations has led to adverse events in the past. In this paper we present VeriCAT, an XAI tool built to help analysts understand the limitations of machine translation so that they can determine whether or not MT text should be verified by a human before being acted upon. Based on conversations with experts and analysts, we identify four design requirements that guide the development of VeriCAT. The VeriCAT system combines a Machine Translation (MT) Quality Estimation (QE) model which predicts the translation quality of sentences translated from Russian into English with the FairSeq MT model with a simple-to-use interface. In the VeriCAT UI sentence-level QE scores are shown alongside MT text. We evaluate VeriCAT with a quantitative user study that measures how the tool impacts users’ ability to identify poor quality MT text. Our evaluation shows VeriCAT significantly improves participants' accuracy in identifying poor quality machine translated text, and that participants assigned to QE scores perform as well as those assigned to ground-truth human-generated quality scores. 



%%\input{sections/design_goals}
%%\input{sections/design_process}
%%
%% The acknowledgments section is defined using the "acks" environment
%% (and NOT an unnumbered section). This ensures the proper
%% identification of the section in the article metadata, and the
%% consistent spelling of the heading.
\begin{acks}
TBD
\end{acks}

%%
%% The next two lines define the bibliography style to be used, and
%% the bibliography file.
\bibliographystyle{ACM-Reference-Format}
\bibliography{main}


\end{document}
\endinput
%%
%% End of file `sample-authordraft.tex'.
